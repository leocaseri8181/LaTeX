\documentclass[11pt,notitlepage]{article}
\usepackage{graphicx}
\usepackage[version=3]{mhchem}
\usepackage[brazilian]{babel}
\usepackage[T1]{fontenc}
\usepackage[utf8]{inputenc}
\usepackage[sfdefault]{quattrocento}
\usepackage{enumitem}
\usepackage{multicol}
\usepackage{titling}
\usepackage{titlesec}
\usepackage{lastpage}
\usepackage{indentfirst}
%\hyphenpenalty=5000
%\tolerance=5000
\setlength{\columnsep}{1cm}
\titleformat{\section}
{\normalfont\fontsize{14}{15}\bfseries}{\thesection}{1em}{}
\setlength{\columnsep}{0.5cm}
\setlength{\columnseprule}{1pt}%
\def\columnseprulecolor{\color{black}}
\usepackage{fancyhdr}
\usepackage{parskip}
\usepackage{float}
\usepackage[left=0.25in,right=0.25in,top=0.25in,bottom=0.25in,paperwidth=7.5in,paperheight=10.62in]{geometry}
\usepackage[a4,center,noinfo]{crop}
\pagestyle{fancy}
\lhead{Caderno de questões - Parte 1}
\rhead{Prova Cognitiva Integrada 1 - M5 (2019/S2)}
%\cfoot{\thepage} Apenas o número da página no centro do rodapé
\cfoot{Página \thepage\ de \pageref{LastPage}}

% Perfeito
\titleformat{\section}
{\normalfont}
{\filright
	%\footnotesize NÃO DESCOMENTE
	\bfseries \enspace Questão \thesection\enspace}
{8pt}
{}
% fim do perfeito

% Perfeito
\titleformat{\part}[frame]
{\normalfont}
{\filright
	%\footnotesize NÃO DESCOMENTE
	\bfseries \enspace Caso Clínico \thepart\enspace}
{8pt}
{}
% fim do perfeito

%\setcounter{section}{x}
\renewcommand{\headrulewidth}{0.4pt}
\renewcommand{\footrulewidth}{0.4pt} 
%\renewcommand{\thesection}{Questão \arabic{section}                                                 

\setlength{\droptitle}{-6em}     % Eliminate the default vertical space
\addtolength{\droptitle}{-6pt}   % Only a guess. Use this for adjustment

\titlespacing{\section}{0pt}{6pt}{6pt}

\title{\vspace{-5ex}}
%\title{Prova Cognitiva Integrada 1 - M6 - 2019/S1 - 05/04/2019 }
\date{\vspace{-5ex}}
\preauthor{}
\postauthor{}
\author{}

%######################
\begin{document}
%######################

\maketitle % habilita a exibição do título do documento
%\thispagestyle{empty} % Limpa definições de estilo e usa aquelas do pacote fancy

\part{Uma paciente de 45 anos de idade chega à Unidade de Pronto Atendimento, agitada, com queixa de dor em cólica em flanco direito do abdome, irradiada para região perineal, acompanhada de disúria e dois episódios de vômitos. Medicada com antiespasmódico e analgésico, teve melhora significativa da dor. O exame clínico revelou punho percussão dolorosa na região lombar direita. O exame da região cervical evidenciou achado incidental de nódulo em triângulo anterior esquerdo, de 1 cm de diâmetro, móvel à deglutição. Indicado exame de urina e sangue.  Durante a coleta de urina para análise, a técnica de laboratório notou hematúria macroscópica. Meses após a mesma paciente, mas nesta oportunidade como acompanhante, leva ao filho de 20 anos de idade a consulta de Urologia por uma massa escrotal observada há 7 meses. O exame físico revela um nódulo de consistência dura no testículo esquerdo, sugestivo câncer de testículo. A massa não é passível de transiluminação, e mostra-se sólida à ultrassonografia.}

\section{Qual a hipótese diagnóstica que você faria diante dos sintomas da paciente?}
\noindent\makebox[\linewidth]{\rule{\textwidth}{0.5pt}}
\noindent\makebox[\linewidth]{\rule{\textwidth}{0.5pt}}
\vspace{0.5cm}

\section{A hematúria macroscópica percebida durante a coleta de urina pode ser de qual origem? Explique.}
\noindent\makebox[\linewidth]{\rule{\textwidth}{0.5pt}}
\noindent\makebox[\linewidth]{\rule{\textwidth}{0.5pt}}
\vspace{0.5cm}

\section{Qual o exame laboratorial para a avaliação da origem da hematúria e sua característica quando a hematúria é de origem glomerular são:}
\begin{multicols}{2}
	\setlength{\columnseprule}{0pt}
	\begin{enumerate}[label=(\alph*)]
		\item urinálise (etapa de sedimentoscopia); bastando a presença de alguns eritrócitos dismórficos (na forma de acantócitos ou em forma de anel com uma ou mais protusões)
		\item urinálise (etapa de sedimentoscopia); presença de mais de 80\% dos eritrócitos isomórficos (sem alterações na forma)
		\item estudo do dismorfismo eritrocitário; presença de mais de 80\% dos eritrócitos dismórficos (na forma de acantócitos ou em forma de anel com uma ou mais protusões)
		\item estudo do dismorfismo eritrocitário; bastando a presença de alguns eritrócitos dismórficos (na forma de acantócitos ou em forma de anel com uma ou mais protusões)
		\item estudo do dismorfismo eritrocitário; presença de mais de 80\% dos eritrócitos isomórficos (sem alterações na forma)
	\end{enumerate}
\end{multicols}
\vspace{0.5cm}

\section{Qual a sua suspeita clínica da localização anatômica do nódulo cervical encontrado? Explique.}
\noindent\makebox[\linewidth]{\rule{\textwidth}{0.5pt}}
\noindent\makebox[\linewidth]{\rule{\textwidth}{0.5pt}}
\vspace{0.5cm}

\section{Quais os diagnósticos diferenciais para o nódulo cervical encontrado e quais características do mesmo você esperaria para cada etiologia possível?}
\noindent\makebox[\linewidth]{\rule{\textwidth}{0.5pt}}
\noindent\makebox[\linewidth]{\rule{\textwidth}{0.5pt}}
\noindent\makebox[\linewidth]{\rule{\textwidth}{0.5pt}}
\vspace{0.5cm}

\section{Considerando a descrição clínica do paciente, qual das seguintes neoplasias é o diagnóstico mais provável?}
\begin{multicols}{2}
	\setlength{\columnseprule}{0pt}
	\begin{enumerate}[label=(\alph*)]
		\item Carcinoma embrionário
		\item Seminoma
		\item Teratocarcinoma
		\item Teratoma maduro
		\item Tumor de células germinativas misto
	\end{enumerate}
\end{multicols}
\vspace{0.5cm}

\section{Quais marcadores tumorais sorológicos são valiosos no diagnostico, tratamento e prognóstico desta neoplasia? }
\begin{multicols}{2}
	\setlength{\columnseprule}{0pt}
	\begin{enumerate}[label=(\alph*)]
		\item Gonadotrofina coriônica humana, citoqueratina 20, vimentina
		\item Lactato desidrogenase, desmina, CD 20
		\item Gonadotrofina coriônica, alfa-actinina, CA 19.9
		\item Antígeno carcinoembrionário, lactato desidrogenase, gonadotrofina coriônica humana
		\item Antígeno carcinoembrionário, calcitonina; proteína da matriz nuclear
	\end{enumerate}
\end{multicols}
\vspace{0.5cm}

\part{Um paciente masculino de 63 anos de idade, negro, aposentado, com história de câncer de próstata metastático há 7 meses baixo cuidados paliativos é trazido ao ambulatório de urologia por desconforto em hipogástrio. Ao exame físico, por meio do toque retal, identificou-se massa pélvica pétrea de superfície irregular. Com o objetivo de avaliar a extensão atual da doença foi indicada cintilografia óssea, a qual revelou múltiplas imagens compatíveis de implantes metastáticos em osso ilíaco esquerdo, e segmento toracolombar (T11 a L3). No último ano apresentou ITU de repetição como resultado de cateterização vesical.}

\section{Qual sistema/pontuação/escala/escore histológico é utilizado pelo patologista na graduação histopatológica do câncer de próstata?}
\begin{multicols}{2}
	\setlength{\columnseprule}{0pt}
	\begin{enumerate}[label=(\alph*)]
		\item Escala de Fisher
		\item Sistema de Gleason
		\item Classificação da Organização Mundial da Saúde para câncer de próstata
		\item Classificação da Working Formulation
		\item Classificação de Hunt \& Hees
	\end{enumerate}
\end{multicols}
\vspace{0.5cm}

\section{Provavelmente a doença metastática descrita e identificada na cintilografia óssea é do tipo:}
\begin{multicols}{2}
	\setlength{\columnseprule}{0pt}
	\begin{enumerate}[label=(\alph*)]
		\item osteolítica
		\item osteoblástica
		\item mista
		\item cancerígena
		\item células gigantes
	\end{enumerate}
\end{multicols}
\vspace{0.5cm}

\section{As prováveis alterações características de ITU encontradas no exame químico e na sedimentoscopia do EAS são:}
\begin{multicols}{2}
	\setlength{\columnseprule}{0pt}
	\begin{enumerate}[label=(\alph*)]
		\item corpos cetônicos: positivo; hematúria, com possibilidade de bacteriúria e cilindros hemáticos.
		\item proteínas: positivo e cilindros leucocitários.
		\item glicose: positivo; hematúria, com possibilidade de bacteriúria e cilindros hemáticos.
		\item nitrito: positivo; leucocitúria, com possibilidade de bacteriúria e cilindros leucocitários
		\item nitrito: positivo; piúria, hematúria, com possibilidade de bacteriúria, cilindros leucocitários e hemáticos.
	\end{enumerate}
\end{multicols}
\vspace{0.5cm}

\part{Homem de 78 anos de idade após queda da própria altura, sofreu traumatismo craniano em região parietal esquerda, sem lesões em couro cabeludo. Chegou inconsciente ao serviço de emergência onde foi realizada uma gasometria arterial e tomografia computadorizada de crânio, revelando um hematoma subdural e realizado o teste de Allen com resultado positivo.}
\vspace{0.5cm}

\section{Considerando essa informação, o traumatismo craniano descrito caracteriza a ação de um instrumento:}
\begin{multicols}{2}
	\setlength{\columnseprule}{0pt}
	\begin{enumerate}[label=(\alph*)]
		\item contuso
		\item contundente
		\item cortocontundente
		\item perfurocontundente
		\item cortante
	\end{enumerate}
\end{multicols}
\vspace{0.5cm}

\section{Interpretando a relação entre o corpo da vítima e o agente causador da lesão, podemos verificar que o mecanismo da lesão foi produzido pela: }
\begin{multicols}{2}
	\setlength{\columnseprule}{0pt}
	\begin{enumerate}[label=(\alph*)]
		\item ação do instrumento sobre o corpo da vítima (meio ativo)
		\item ação do corpo da vítima sobre o instrumento (meio passivo)
		\item forma mista
	\end{enumerate}
\end{multicols}
\vspace{0.5cm}

\section{Quais as principais artérias para realização do teste realizado?}
\begin{multicols}{2}
	\setlength{\columnseprule}{0pt}
	\begin{enumerate}[label=(\alph*)]
		\item femural e pediosa
		\item radial e braquial
		\item braquial e femural
		\item radial e ulnar
		\item femural e radial
	\end{enumerate}
\end{multicols}
\vspace{0.5cm}

\section{Descreva os tipos de resultados do teste realizado.}
\noindent\makebox[\linewidth]{\rule{\textwidth}{0.5pt}}
\noindent\makebox[\linewidth]{\rule{\textwidth}{0.5pt}}
\noindent\makebox[\linewidth]{\rule{\textwidth}{0.5pt}}
\noindent\makebox[\linewidth]{\rule{\textwidth}{0.5pt}}
\noindent\makebox[\linewidth]{\rule{\textwidth}{0.5pt}}
\noindent\makebox[\linewidth]{\rule{\textwidth}{0.5pt}}
%\vspace{0.5cm}

\part{H.M.J. sexo masculino, 14 anos, 42,2 kg, 1,52 m de altura, refere que, há 10 meses, começou a apresentar quadro de edema de membros inferiores e de face, associado a urina escura e espumosa e cansaço aos médios esforços. Na ocasião, procurou assistência médica no serviço público de referência de sua cidade e, após consulta  com um clínico geral, foi encaminhado para um nefrologista, que, após internação  e obtenção dos resultados dos exames complementares com resultados: Colesterol Total: 515 mg/dL, LDL: 362 mg/dL, HDL: 44 mg/dL, Triglicérides: 715 mg/dL, Ureia: 69 mg/dL, Creatinina: 2,64 mg/dL, proteinúria de 189 mg/kg/24h, iniciou politerapia medicamentosa. H.M.J. evoluiu com desaparecimento dos edemas de membros inferiores e face e melhora da função renal. No 8º dia de internação apresentava-se assintomático. Em contra referência, já na UBS/ESF, este jovem comparece, acompanhado de sua namorada, I.O.C., de 16 anos, também usuária desta unidade, solicitando, em consulta com o médico da família, um anticoncepcional, via oral, para a namorada, que demonstrou concordância e interesse em estar ali, já que ambos mantém relação sexual frequente e querem orientações quanto ao uso de preservativos e  outros métodos para que ela não fique grávida. }
\vspace{0.5cm}

\section{Nesta situação, a conduta mais assertiva deste médico da família, seguindo a ética médica seria:}
\begin{multicols}{2}
	\setlength{\columnseprule}{0pt}
	\begin{enumerate}[label=(\alph*)]
		\item dizer que só pode prescrever o método pedido após falar com um dos pais ou responsável legal de cada parte, já que ambos são legalmente incapazes, mas, principalmente, da menina adolescente, para que ao menos sua mãe/responsável legal fique ciente da vida sexual ativa dela. 
		\item Prescrever a pílula anticoncepcional, mas deixando claro a necessidade de uso de preservativo e convocar os pais ou responsáveis legais para ficarem cientes da situação.
		\item não informar os pais ou responsáveis legais, já que não tem esta obrigação para tal, mas não prescrever o anticoncepcional e orientar somente o menino quanto ao uso do preservativo masculino. 
		\item prescrever o anticoncepcional mais adequado a ela e dar orientações a ambos sobre o uso dos métodos existentes, já que neste caso não é considerado infração ética, pois mesmo sendo menores de idade, os dois mostram-se capazes de decidir sobre estes assuntos, não havendo necessidade de comunicação aos pais/responsáveis.
	\end{enumerate}
\end{multicols}
\vspace{0.5cm}

\section{Assinale a alternativa que relaciona a possível terapia medicamentosa, administrada em associação, para o tratamento de A.J.C. }
\begin{multicols}{2}
	\setlength{\columnseprule}{0pt}
	\begin{enumerate}[label=(\alph*)]
		\item Espironolactona, losartana e atorvastatina
		\item Prednisona, losartana sinvastatina/ezetimiba
		\item Losartana, prednisona, fibrato e atorvastatina
		\item Losartana, espironolactona, furosemida e atorvastatina
		\item Prednisona, espironolactona, furosemida e atorvastatina
	\end{enumerate}
\end{multicols}
\vspace{0.5cm}

\section{Justifique farmacologicamente a alternativa corretamente assinalada}
\noindent\makebox[\linewidth]{\rule{\textwidth}{0.5pt}}
\noindent\makebox[\linewidth]{\rule{\textwidth}{0.5pt}}
\noindent\makebox[\linewidth]{\rule{\textwidth}{0.5pt}}
\noindent\makebox[\linewidth]{\rule{\textwidth}{0.5pt}}
\noindent\makebox[\linewidth]{\rule{\textwidth}{0.5pt}}
\noindent\makebox[\linewidth]{\rule{\textwidth}{0.5pt}}
\noindent\makebox[\linewidth]{\rule{\textwidth}{0.5pt}}
\noindent\makebox[\linewidth]{\rule{\textwidth}{0.5pt}}
\noindent\makebox[\linewidth]{\rule{\textwidth}{0.5pt}}
\noindent\makebox[\linewidth]{\rule{\textwidth}{0.5pt}}
\vspace{0.5cm}

\section{Quais efeitos são esperados de interação entre os fármacos contidos na terapia proposta na alternativa corretamente assinalada. }
\noindent\makebox[\linewidth]{\rule{\textwidth}{0.5pt}}
\noindent\makebox[\linewidth]{\rule{\textwidth}{0.5pt}}
\noindent\makebox[\linewidth]{\rule{\textwidth}{0.5pt}}
\noindent\makebox[\linewidth]{\rule{\textwidth}{0.5pt}}
\noindent\makebox[\linewidth]{\rule{\textwidth}{0.5pt}}
\noindent\makebox[\linewidth]{\rule{\textwidth}{0.5pt}}
\noindent\makebox[\linewidth]{\rule{\textwidth}{0.5pt}}
\noindent\makebox[\linewidth]{\rule{\textwidth}{0.5pt}}
\noindent\makebox[\linewidth]{\rule{\textwidth}{0.5pt}}
\noindent\makebox[\linewidth]{\rule{\textwidth}{0.5pt}}
\vspace{0.5cm}

\section{Quais efeitos podem ocorrer, em virtude da terapia medicamentosa associada, utilizada para o tratamento de A.J.C.?}
\begin{multicols}{2}
	\setlength{\columnseprule}{0pt}
	\begin{enumerate}[label=(\alph*)]
		\item Miopatia, ginecomastia, hiponatremia, elevação das provas de função hepática
		\item Hipercalemia, hipernatremia, elevação das provas de função hepática
		\item Constipação intestinal, elevação das provas de função hepática, hipomagnesemia
		\item Hipotensão de primeira dose, hipernatremia, hipocalemia e elevação das provas de função hepática
		\item Diarreia, náuseas, hipornatremia, hipercalemia e leucopenia
	\end{enumerate}
\end{multicols}
\vspace{0.5cm}

\part{J.F.S., 28 anos, feminina, branca, casada, advogada, natural de São Carlos-SP, procedente de Araraquara-SP, gestante de 39 semanas completas deu entrada na emergência da Santa Casa, trazida pelo marido, com história de falta de ar há 30 minutos. Informante refere que a mesma é asmática desde à infância e que durante esta gravidez apresentou 03 episódios de crise asmática, ficando internada em dois deles (SIC). Refere que a mesma se encontrava gripada há 04 dias, apresentando episódios de febre não aferida (SIC). Refere que hoje após o jantar, iniciou com quadro de falta de ar, chiado no peito, fraqueza, dor tipo cólica em baixo ventre de forte intensidade e saída de secreção sanguinolenta pela vagina (SIC). Fez uso de medicamento inalatório (bombinha), porém sem melhora do quadro. Apesar da falta de ar, informa que já estão acostumados com suas crises, porém, o que realmente os preocupou foram as contrações e o sangramento (SIC). A gestante refere ainda que as dores aumentaram de intensidade e de frequência, ficando quase insuportáveis.  APH: Primigesta, asmática, nega tabagismo e/ou etilismo, fez o pré-natal no Posto de Saúde (08 consultas). AF: Pai hipertenso, mãe asmática. Exame Físico: Regular estado geral, dispneica, afebril (Tax 36,5°C), mucosas coradas e hidratadas, anictérica, acianótica. PA: 110 X 70 mmHg    FC: 102 bpm      FR : 24 ipm.   Ritmo cardíaco regular em dois tempos, bulhas normofonéticas, sem sopros. Sem estase jugular. AR: Murmúrio vesicular e frêmito toracovocal presentes, porém levemente diminuídos à direita, discretos sibilos bilaterais sem estertorações, 16 incursões respiratórias por minuto. Idade Gestacional: 39 semanas e 03 dias (confirmadas pela DUM e pelo US do 1º trimestre). Abdome: Gravídico, ruídos hidroaéreos presentes, Altura Uterina de 34 cm, Dorso à esquerda, apresentação cefálica encaixada, variedade de apresentação = O.E.A. BCF= 148 bpm, dinâmica uterina = 3 contrações/ 50 segundos/ 10 minutos. Toque vaginal: colo apagado 90\% , dilatado 3 cm, apresentação cefálica, encaixada, plano +1 de DeLee, bolsa íntegra, perda de tampão mucoso em dedo de luva.}
\vspace{0.5cm}

\section{Em relação ao número de consultas de pré-natal, segundo à OMS, você consideraria o pré-natal da Sra. Joana eficaz? Justifique:}
\noindent\makebox[\linewidth]{\rule{\textwidth}{0.5pt}}
\noindent\makebox[\linewidth]{\rule{\textwidth}{0.5pt}}
\noindent\makebox[\linewidth]{\rule{\textwidth}{0.5pt}}
\vspace{0.5cm}

\section{Explique o significado da variedade de apresentação encontrada no exame físico obstétrico.}
\noindent\makebox[\linewidth]{\rule{\textwidth}{0.5pt}}
\noindent\makebox[\linewidth]{\rule{\textwidth}{0.5pt}}
\noindent\makebox[\linewidth]{\rule{\textwidth}{0.5pt}}
\vspace{0.5cm}

\section{Você internaria a Sra. Joana? Sim ou não. Justifique.}
\noindent\makebox[\linewidth]{\rule{\textwidth}{0.5pt}}
\noindent\makebox[\linewidth]{\rule{\textwidth}{0.5pt}}
\noindent\makebox[\linewidth]{\rule{\textwidth}{0.5pt}}
\vspace{0.5cm}

\section{Com relação ao exame físico obstétrico, quais os achados que justificam a Sra. Joana estar em Trabalho de Parto?}
\noindent\makebox[\linewidth]{\rule{\textwidth}{0.5pt}}
\noindent\makebox[\linewidth]{\rule{\textwidth}{0.5pt}}
\vspace{0.5cm}

\section{Com relação à assistência ao parto, quais as medidas que podem ser adotadas para ajudar a Sra. Joana, neste período?}
\noindent\makebox[\linewidth]{\rule{\textwidth}{0.5pt}}
\noindent\makebox[\linewidth]{\rule{\textwidth}{0.5pt}}
\noindent\makebox[\linewidth]{\rule{\textwidth}{0.5pt}}
\vspace{0.5cm}

\section{Calcule o valor da PAM da paciente. }
\noindent\makebox[\linewidth]{\rule{\textwidth}{0.5pt}}
\noindent\makebox[\linewidth]{\rule{\textwidth}{0.5pt}}
\noindent\makebox[\linewidth]{\rule{\textwidth}{0.5pt}}
\noindent\makebox[\linewidth]{\rule{\textwidth}{0.5pt}}
\vspace{0.5cm}

\section{Descreva três (3) cuidados fundamentais para verificar corretamente o pulso periférico desta paciente.}
\noindent\makebox[\linewidth]{\rule{\textwidth}{0.5pt}}
\noindent\makebox[\linewidth]{\rule{\textwidth}{0.5pt}}
\noindent\makebox[\linewidth]{\rule{\textwidth}{0.5pt}}
\noindent\makebox[\linewidth]{\rule{\textwidth}{0.5pt}}
\noindent\makebox[\linewidth]{\rule{\textwidth}{0.5pt}}
\noindent\makebox[\linewidth]{\rule{\textwidth}{0.5pt}}
\vspace{0.5cm}

%###############################################################################
% Final do documento
\end{document}
%####################