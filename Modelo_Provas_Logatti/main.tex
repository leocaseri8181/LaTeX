\documentclass[11pt]{article}
\usepackage{etoolbox}
\usepackage{graphicx}
\usepackage[version=3]{mhchem}
\usepackage[brazil]{babel}
\usepackage[T1]{fontenc}
\usepackage[utf8]{inputenc}
\usepackage[sfdefault]{quattrocento}
\usepackage{enumitem}
\usepackage{multicol}
\usepackage{tabularx}
\usepackage{titlesec}
\usepackage{titling}
\usepackage{blindtext}
\usepackage{lineno}
\usepackage{lipsum}
\usepackage{lastpage}
\titleformat{\section}[frame]
{\filright \normalfont\fontsize{14}{15}}{\thesection}{1em}{}
\setlength{\columnsep}{0.5cm}
%\setlength{\columnseprule}{1pt}
\def\columnseprulecolor{\color{black}}
\usepackage{fancyhdr}
\fancyhf{}
%\rfoot{\thepage \hspace{1pt}/\pageref{LastPage}}
\usepackage{parskip}
\usepackage{float}
\usepackage[left=0.5in,right=0.5in,top=0.5in,bottom=0.5in,paperwidth=7.5in,paperheight=11.12in]{geometry}
\usepackage[a4,center,noinfo]{crop}
\pagestyle{fancy}
\lhead{Avaliação de Química - 1EM - 14/ago/2019}
\rhead{Prof. João Henrique}
\cfoot{Página \thepage\ de \pageref{LastPage}}

\setlength{\droptitle}{-4em}     % Eliminate the default vertical space
\addtolength{\droptitle}{-4pt}   % Only a guess. Use this for adjustment


%\setcounter{section}{25}
\renewcommand{\headrulewidth}{0.4pt}
\renewcommand{\footrulewidth}{0.4pt} 
\renewcommand{\thesection}{Questão \arabic{section} }

%\title{Prova Cognitiva Integrada 1 - Módulo 6 - 2019/S1 }
\title{\vspace{-5ex}}

\date{{\vspace{-5ex}}}
\preauthor{}
\postauthor{}
\author{}

\makeatletter
\let\ps@mystyle\ps@plain % Copy the plain style to mystyle
\patchcmd{\ps@mystyle}{\thepage}{Página\ \thepage\ de\ \pageref{LastPage}}{}{} % replace number by desired string
\makeatother
\pagestyle{mystyle}

%######################
\begin{document}
%######################
\maketitle\thispagestyle{mystyle}
%\maketitle % habilita a exibição do título do documento
%\thispagestyle{empty} % Limpa definições de estilo e usa aquelas do pacote fancy

\begin{figure}[hbt]
	%\includegraphics[scale=0.5]{logatti.jpg}
	%\caption{}
\end{figure}

\begin{center}\textbf{Avaliação de Sistemas Distribuídos - Prof. João Henrique - 27/jun/2019}
\end{center}

\begin{tabularx}{468pt}{|X|X|c|}
\hline
\textbf{Nome} & \textbf{Curso} & \textbf{Código}\\
	\hline
	& & \\   
	\hline
\end{tabularx}

\section{Apresente uma diferenciação rápida entre os conceitos programa e "processo".}

\section{Usando no máximo 8 linhas de texto, descreva os três estados possíveis para um processo na memória do computador.}
\begin{linenumbers}
\resetlinenumber
\noindent\makebox[\linewidth]{\rule{\textwidth}{0.5pt}}
\noindent\makebox[\linewidth]{\rule{\textwidth}{0.5pt}}
\noindent\makebox[\linewidth]{\rule{\textwidth}{0.5pt}}
\noindent\makebox[\linewidth]{\rule{\textwidth}{0.5pt}}
\noindent\makebox[\linewidth]{\rule{\textwidth}{0.5pt}}
\noindent\makebox[\linewidth]{\rule{\textwidth}{0.5pt}}
\noindent\makebox[\linewidth]{\rule{\textwidth}{0.5pt}}
\noindent\makebox[\linewidth]{\rule{\textwidth}{0.5pt}}
\end{linenumbers}

\section{Um conceito de grande importância em Sistemas Operacionais é conhecido como "Região Crítica". Explique esse conceito.}
\begin{linenumbers}
\resetlinenumber
\noindent\makebox[\linewidth]{\rule{\textwidth}{0.5pt}}
\noindent\makebox[\linewidth]{\rule{\textwidth}{0.5pt}}
\noindent\makebox[\linewidth]{\rule{\textwidth}{0.5pt}}
\noindent\makebox[\linewidth]{\rule{\textwidth}{0.5pt}} \noindent\makebox[\linewidth]{\rule{\textwidth}{0.5pt}}
\end{linenumbers}

\section{Caso não seja evitada a chamada "condição de disputa", pode ocorrer um problema indesejável envolvendo dois usuários que tentam imprimir um arquivo em uma impressora, mediante o uso de um spool (simultaneous peripheral operations on-line). Descreva esse problema.}
\begin{linenumbers}
\resetlinenumber
\noindent\makebox[\linewidth]{\rule{\textwidth}{0.5pt}}
\noindent\makebox[\linewidth]{\rule{\textwidth}{0.5pt}}
\noindent\makebox[\linewidth]{\rule{\textwidth}{0.5pt}}
\noindent\makebox[\linewidth]{\rule{\textwidth}{0.5pt}}
\noindent\makebox[\linewidth]{\rule{\textwidth}{0.5pt}}
\noindent\makebox[\linewidth]{\rule{\textwidth}{0.5pt}}
\noindent\makebox[\linewidth]{\rule{\textwidth}{0.5pt}}
\noindent\makebox[\linewidth]{\rule{\textwidth}{0.5pt}}
\end{linenumbers}

\section{Explique o conceito de "thread" usando um editor de textos comercial, como o Microsoft Word.}
\begin{linenumbers}
\resetlinenumber
\noindent\makebox[\linewidth]{\rule{\textwidth}{0.5pt}}
\noindent\makebox[\linewidth]{\rule{\textwidth}{0.5pt}}
\noindent\makebox[\linewidth]{\rule{\textwidth}{0.5pt}}
\noindent\makebox[\linewidth]{\rule{\textwidth}{0.5pt}}
\noindent\makebox[\linewidth]{\rule{\textwidth}{0.5pt}}
\end{linenumbers}

\section{As expressões "fragmentação interna" e "fragmentação externa" estão relacionadas com o gerenciamento de memória. Diferencie os dois termos.}
\begin{linenumbers}
\resetlinenumber
\noindent\makebox[\linewidth]{\rule{\textwidth}{0.5pt}}
\noindent\makebox[\linewidth]{\rule{\textwidth}{0.5pt}}
\noindent\makebox[\linewidth]{\rule{\textwidth}{0.5pt}}
\noindent\makebox[\linewidth]{\rule{\textwidth}{0.5pt}}
\noindent\makebox[\linewidth]{\rule{\textwidth}{0.5pt}}
\end{linenumbers}

\section{Como os termos "página" e "quadro" estão relacionados?}
\begin{linenumbers}
\resetlinenumber
\noindent\makebox[\linewidth]{\rule{\textwidth}{0.5pt}}
\noindent\makebox[\linewidth]{\rule{\textwidth}{0.5pt}}
\noindent\makebox[\linewidth]{\rule{\textwidth}{0.5pt}}
\noindent\makebox[\linewidth]{\rule{\textwidth}{0.5pt}}
\noindent\makebox[\linewidth]{\rule{\textwidth}{0.5pt}}
\end{linenumbers}

\section{Um sistema de arquivos com suporte a journaling pode ser muito resistente a desligamento indevido do sistema operacional. Como isso ocorre?}
\begin{linenumbers}
\resetlinenumber
\noindent\makebox[\linewidth]{\rule{\textwidth}{0.5pt}}
\noindent\makebox[\linewidth]{\rule{\textwidth}{0.5pt}}
\noindent\makebox[\linewidth]{\rule{\textwidth}{0.5pt}}
\noindent\makebox[\linewidth]{\rule{\textwidth}{0.5pt}}
\noindent\makebox[\linewidth]{\rule{\textwidth}{0.5pt}}
\end{linenumbers}

\section{Um sistema de arquivo muito usado em distribuições Linux é conhecido como "ext4" e apresenta a possibilidade de restingir o acesso a um dado recurso (arquivo ou diretório) de forma muito granular. Relacione essa informação com a codificação "rwx" e números como "750".}
\begin{linenumbers}
\resetlinenumber
\noindent\makebox[\linewidth]{\rule{\textwidth}{0.5pt}}
\noindent\makebox[\linewidth]{\rule{\textwidth}{0.5pt}}
\noindent\makebox[\linewidth]{\rule{\textwidth}{0.5pt}}
\noindent\makebox[\linewidth]{\rule{\textwidth}{0.5pt}}
\noindent\makebox[\linewidth]{\rule{\textwidth}{0.5pt}}
\noindent\makebox[\linewidth]{\rule{\textwidth}{0.5pt}}
\noindent\makebox[\linewidth]{\rule{\textwidth}{0.5pt}}
\noindent\makebox[\linewidth]{\rule{\textwidth}{0.5pt}}
\noindent\makebox[\linewidth]{\rule{\textwidth}{0.5pt}}
\noindent\makebox[\linewidth]{\rule{\textwidth}{0.5pt}}
\end{linenumbers}

\section{Defina, em detalhes, o funcionamento dos sistemas RAID 0, 1 e 5}
\begin{linenumbers}
\resetlinenumber
\noindent\makebox[\linewidth]{\rule{\textwidth}{0.5pt}}
\noindent\makebox[\linewidth]{\rule{\textwidth}{0.5pt}}
\noindent\makebox[\linewidth]{\rule{\textwidth}{0.5pt}}
\noindent\makebox[\linewidth]{\rule{\textwidth}{0.5pt}}
\noindent\makebox[\linewidth]{\rule{\textwidth}{0.5pt}}
\noindent\makebox[\linewidth]{\rule{\textwidth}{0.5pt}}
\noindent\makebox[\linewidth]{\rule{\textwidth}{0.5pt}}
\noindent\makebox[\linewidth]{\rule{\textwidth}{0.5pt}}
\noindent\makebox[\linewidth]{\rule{\textwidth}{0.5pt}}
\noindent\makebox[\linewidth]{\rule{\textwidth}{0.5pt}}
\end{linenumbers}

%###############################################################################
% Final do documento
\end{document}
%####################